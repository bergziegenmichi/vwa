\documentclass[12pt, ngerman]{article}
\usepackage{tikz}
\usepackage{graphicx}
\usepackage[utf8]{inputenc}
\usepackage[authordate, backend=biber]{biblatex-chicago}
\usepackage[ngerman]{babel}
\usepackage{babel}
\usepackage{eso-pic}
\usepackage[a4paper, lmargin=3.5cm, rmargin=2cm]{geometry}
\usepackage{hyperref}
\hypersetup{colorlinks=true, allcolors=black}
%Allow editing of TOF and TOT
\usepackage[titles]{tocloft}
\usepackage[none]{tocbibind}

\usepackage{etoolbox}
\usepackage{csquotes}
\usepackage{float}
\usepackage{amsmath}
\usepackage{caption}
\usepackage[parfill]{parskip}

\usepackage{mhchem}

\usepackage{subfiles}

\setcounter{secnumdepth}{4}
\setcounter{tocdepth}{4}

% make Abbildungsverzeichnis have a section number
%\renewcommand{\listoffigures}{\begingroup
%	\tocsection
%	\tocfile{\listfigurename}{lof}
%\endgroup}

% Hinzufügen der Quellen
% Fuer eine einfachere Verwendung ist es empfehlenswert diese Datei mithilfe
% von Zotero & /betterbiblatex auf dem neuesten Stand zu halten
\addbibresource{Quellen.bib}

\begin{document}
% Einfügen des Titelblatts
% Um das Titelblatt zu editieren, müssen in
% titel.tex die entsprechenden Werte geändert werden

\newcommand{\VWAtitel}{"`Nicht nur sauber, sondern rein"' - Waschmittel und ihre Inhaltsstoffe}
\newcommand{\VWAauthor}{Hofmann Michael}
\newcommand{\klasse}{8C}
\newcommand{\jahr}{2022/23}
\newcommand{\betreuungslehrperson}{Mag. Barbara Hirss}
\newcommand{\vorlagedatum}{tt.mm.jjjj}
\include{template.tex}

% Nummerierung zuerst mit roemischen Zahlen, danach mit
% arabischen
\pagenumbering{Roman}
\setcounter{page}{2}

\abstract
...
%\clearpage
\tableofcontents
\clearpage
\pagenumbering{arabic}

\section{Einleitung}
Warum Waschmittel wichtig sind und paar Fakten zu Waschmitteln generell. Auch wie sich Waschmittel entwickelt haben.

\section{Inhaltsstoffe von Waschmitteln}

\subfile{tenside.tex}

\subsection{Bleichmittel}

\subsubsection{Sauerstoffbasis}

\subsubsection{Chlor Basis}

\subsubsection{Bleichaktivatoren}

\subsubsection{Bleichkatalysatoren}

\subsubsection{Percarbonsäuren (weiß nicht ob das so wichtig ist)}

\subfile{enzyme.tex}

\subsection{Sonstige Inhaltsstoffe}
Optische Aufheller
Vergrauungs-, Verfärbungs- und Schauminhibitoren. Farb- und Duftstoffe.


\section{Probleme mit Waschmitteln}
Auswirkungen auf die Umwelt, hormonell wirksame Stoffe, Hautverträglichkeit

\section{Ausgewählte Stoffe}
Noch eine besser Überschrift auf alle Fälle.\\

Häufig verwendete Stoffe; Stoffe die noch verwendet verwendet werden, bei denen es aber bedenken gibt; "`neue"' Stoffe die in der Zukunft interessant werden könnten\\

Diese Stoffe beschreiben, je nach Relevanz Herstellung, Verwendung, Probleme, etc






% TOC und Bibliography müssen auf eigenen Seiten sein
\clearpage
\nocite{*}
\printbibliography[title=\section{Literaturverzeichnis}]
%\listoffigures
%\listoftables

\end{document}

