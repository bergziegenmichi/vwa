\documentclass[12pt, ngerman]{article}
\usepackage{tikz}
\usepackage{graphicx}
\usepackage[utf8]{inputenc}
\usepackage[authordate, backend=biber]{biblatex-chicago}
\usepackage[ngerman]{babel}
\usepackage{babel}
\usepackage{eso-pic}
\usepackage[a4paper, lmargin=3.5cm, rmargin=2cm]{geometry}
\usepackage{hyperref}
\hypersetup{colorlinks=true, allcolors=black}
%Allow editing of TOF and TOT
\usepackage[titles]{tocloft}
\usepackage[none]{tocbibind}

\usepackage{etoolbox}
\usepackage{csquotes}
\usepackage{float}
\usepackage{amsmath}
\usepackage{caption}
\usepackage[parfill]{parskip}


% make Abbildungsverzeichnis have a section number
%\renewcommand{\listoffigures}{\begingroup
%	\tocsection
%	\tocfile{\listfigurename}{lof}
%\endgroup}

% Hinzufügen der Quellen
% Fuer eine einfachere Verwendung ist es empfehlenswert diese Datei mithilfe
% von Zotero & /betterbiblatex auf dem neuesten Stand zu halten
\addbibresource{Quellen.bib}

\begin{document}
% Einfügen des Titelblatts
% Um das Titelblatt zu editieren, müssen in
% titel.tex die entsprechenden Werte geändert werden

\newcommand{\VWAtitel}{"`Nicht nur sauber, sondern rein"' - Waschmittel und ihre Inhaltsstoffe}
\newcommand{\VWAauthor}{Hofmann Michael}
\newcommand{\klasse}{8C}
\newcommand{\jahr}{2022/23}
\newcommand{\betreuungslehrperson}{Mag. Barbara Hirss}
\newcommand{\vorlagedatum}{tt.mm.jjjj}
\include{template.tex}

% Nummerierung zuerst mit roemischen Zahlen, danach mit
% arabischen
\pagenumbering{Roman}
\setcounter{page}{2}

\abstract
...
%\clearpage
\tableofcontents
\clearpage
\pagenumbering{arabic}

\section{Einleitung}
Warum Waschmittel wichtig sind und paar Fakten zu Waschmitteln generell. Auch wie sich Waschmittel entwickelt haben.

\section{Inhaltsstoffe von Waschmitteln}


\subsection{Tenside}

Tenside sind, aufgrund ihrer Fähigkeit unpolare Stoffe zu lösen, die wichtigsten Inhaltsstoffe von Waschmitteln. Sie sind amphiphil, das bedeutet sie weisen sowohl einen hydrophilen, als auch einen lipophilen Molekülteil auf. Der immer ähnlich aussehende, lipophile Teil wird bei allen Tensiden durch einen langkettigen Kohlenwasserstoffrest gebildet. Der hydrophile Teil hingegen kann sehr unterschiedlich aufgebaut sein, wodurch Tenside, je nach Ladung des Molekülteils in vier Klassen eingeteilt werden können:
\begin{itemize}
	\item Anionische Tenside
	\item Kationische Tenside
	\item Nichtionische Tenside
	\item Amphotere Tenside
\end{itemize}

Der genaue Aufbau des hydrophilen Teils und die Struktur des lipophilen Alkyl-Restes beeinflussen die Waschaktivität der Tenside. Durch die Anwesenheit von gelösten Ionen können schwerlösliche Salze entstehen, welche das Waschvermögen einschränken oder sich auf der Wäsche ablagern können.

Das bekannteste Tensid ist die Seife, welche zur Gruppe der anionischen Tenside gehört.

\subsubsection{Anionische Tenside}

\subsubsection{Kationische Tenside}

\subsubsection{Nichtionische Tenside}

\subsubsection{Amphotere Tenside}

\subsection{Gerüststoffe}

\subsection{Bleichmittel}

\subsubsection{Sauerstoffbasis}

\subsubsection{Chlor Basis}

\subsubsection{Bleichaktivatoren}

\subsubsection{Bleichkatalysatoren}

\subsubsection{Percarbonsäuren (weiß nicht ob das so wichtig ist)}

\subsection{Enzyme}

sind in modernen Waschmitteln wichtig weil energiesparend

Proteasen (Eiweiß)

Amylasen (Kohlenhydrate)

Lipasen ( Fette, wie Kosmetika, Schmalz, Talg)

Cellulasen (geschädigte Cellulosefasern)

Mannanasen (Verdicker, idk ob wichtig)

\subsection{Sonstige Inhaltsstoffe}
Optische Aufheller
Vergrauungs-, Verfärbungs- und Schauminhibitoren. Farb- und Duftstoffe.


\section{Probleme mit Waschmitteln}
Auswirkungen auf die Umwelt, hormonell wirksame Stoffe, Hautverträglichkeit

\section{Ausgewählte Stoffe}
Noch eine besser Überschrift auf alle Fälle.\\

Häufig verwendete Stoffe; Stoffe die noch verwendet verwendet werden, bei denen es aber bedenken gibt; "`neue"' Stoffe die in der Zukunft interessant werden könnten\\

Diese Stoffe beschreiben, je nach Relevanz Herstellung, Verwendung, Probleme, etc






% TOC und Bibliography müssen auf eigenen Seiten sein
\clearpage
\nocite{*}
\printbibliography[title=\section{Literaturverzeichnis}]
%\listoffigures
%\listoftables

\end{document}

