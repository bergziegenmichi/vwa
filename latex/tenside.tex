\documentclass[main.tex]{subfiles}

\begin{document}

\subsection{Tenside}

Tenside (lat.: tensio = Spannung) sind, aufgrund ihrer Fähigkeit unpolare Stoffe zu lösen, die wichtigsten Inhaltsstoffe von Waschmitteln. Sie sind amphiphil, das bedeutet sie weisen sowohl einen hydrophilen, als auch einen lipophilen Molekülteil auf. Tenside können als Kopf-Schwanz-Modell dargestellt werden. Der lipophile Schwanz, wird immer durch einen Kohlenwasserstoffrest gebildet und ist immer ähnlich aufgebaut. Der hydrophile Kopf hingegen zeigt deutlich mehr Variationen, wodurch Tenside, je nach Ladung des Kopfes, in vier Klassen eingeteilt:

\begin{itemize}
	\item Anionische Tenside
	\item Kationische Tenside
	\item Nichtionische Tenside
	\item Amphotere Tenside
\end{itemize}

Der genaue Aufbau des hydrophilen Teils und die Struktur des lipophilen Alkyl-Restes beeinflussen die Waschaktivität der Tenside. Durch die Anwesenheit von gelösten Ionen können schwerlösliche Salze entstehen, welche das Waschvermögen einschränken oder sich auf der Wäsche ablagern können. In modernen Waschmitteln werden daher Gerüststoffe verwendet, welche Erdalkalimetalle aus dem Wasser filtern sollen. Außerdem werden Tenside verwendet, welche unempfindlich im Bezug zur Wasserhärte sind. Jedoch haben auch die Temperatur, der pH-Wert oder andere Inhaltsstoffe einen Einfluss auf die Waschkraft.

\subsubsection{Funktionsweise der Tenside}

Durch ihren amphiphilen Aufbau lagern sich Tenside hauptsächlich an Grenzflächen ab. Unter einer Grenzfläche versteht man die Grenze zwischen zweier nicht mischbarer Phasen. Durch die hohe Grenzflächenaktivität haben sie in wässriger Lösung folgende Eigenschaften:

\begin{itemize}
	\item Sie senken die Oberflächenspannung
	\item Sie bilden Mizellen
	\item Sie ermöglichen das Benetzen hydrophober Flächen
	\item Sie besitzen ein Schaumvermögen (anders Formulieren)
\end{itemize}

Beim Einsatz als Waschmittel können Tenside somit sowohl Schmutz ablösen \\ (Primärwaschvermögen), als auch die erneute Ablagerung von Schmutz verhindern (Sekundärwaschvermögen).

\subsubsection{Tensid-Arten (anders nennen)}

\paragraph{Anionische Tenside}

Bein anionischen Tensiden ist der hydrophile Kopf negativ geladen. Die wichtigsten (funktionellen / hydrophilen) Gruppen sind:

\begin{itemize}
	\item Carboxylat-Anion \ce{R\bond{-}COO-}
	\item Sulfonat-Anion \ce{R\bond{-}SO3-}
	\item Sulfat-Anion \ce{R\bond{-}O-SO3-} oder \ce{R\bond{-}SO4-}
\end{itemize}

Anionische Tenside werden vorwiegend in Form von Natriumsalzen verwendet. Doch auch Kalium, Ammonium oder Triethanolammonium (\ce{(CH2CH2OH)3NH+}) werden als Kationen verwendet. Die Ammonium- und Triethanolammoniumsalze sind wesentlich besser in Wasser löslich, jedoch ist ihre chemische Stabilität in Formulierungen nicht so ausgeprägt (?). Außerdem sind die Natrium- und Kaliumsalze leichter biologisch abbaubar.

Das bekanntest und historisch wichtigste anionische Tenside ist die Seife, auf welche im Abschnitt "`ausgewählte Stoffe"' näher eingegangen wird.

Weitere bedeutende anionische Tenside sind die Alkylbenzolsulfonate, Alkansolfonate, Fettalkoholsulfonate und Fettalkoholethersulfate.

Eventuell näher auf die 4 gruppen eingehen

\paragraph{Kationische Tenside}

Bei kationischen Tensiden ist hydrophile Kopf positiv geladen. Die Ladung kommt von einer stickstoffhaltigen Gruppe (R4N+). Als Kationen werden Chlorid oder Methylsulfat (\ce{CH3OSO3-}) (\ce{CH3O4S-}). Da sich die kationischen Tenside von quartären Ammoniumverbindungen ableiten, werden sie oft als Quats oder QAV bezeichnet.

Kationische Tenside werden als Weichspülmittel verwendet, da sie negativ geladene Flächen belegen können und so eine Oberfläche "`hydrophobieren"' können. NOCH MEHR RECHERCHIEREN

\paragraph{Nichtionische Tenside}

Bei nichtionischen Tensiden ist der hydrophile Kopf nicht geladen. Die hydrophile Eigenschaft wird durch stark polare Gruppen erreicht. Die wichtigsten dieser Gruppen sind Ether- und Hydroxylgruppen, welche auch oft kombiniert werden. Da polare Gruppen eine viel geringere hydrophile Potenz besitzen, als eine ionische Gruppe, ist eine größere Anzahl an polarer Gruppen notwendig.

Gegenüber ionischen Tensiden haben nichtionische Tenside folgende besondere Eigenschaften:

\begin{itemize}
	\item Es finden keine elektrostatischen Wechselwirkungen mit dem Schmutz und den Textilfasern statt.
	\item Durch gezielte Synthese können die Eigenschaften besser kontrolliert und verändert werden.
	\item Bestimmte nichtionische Tenside weisen eine Löslichkeitsanomalie auf, das heißt eine geringere Löslichkeit bei höheren Temperaturen.
\end{itemize}

Wichtige Eigenschaften:
Absolute Hårteunempfindlichkeit;
x Sehr gute Wasch- und Entfettungswirkung bei niedriger Temperatur;
x Schaumarme Tenside;
x Niedrige kritische Micellbildungskonzentration, dadurch ist eine niedrigere Do-
sierung mÇglich;
x Gute Vergrauungsinhibierung bei Synthesefasern.

Die wichtigsten nichtionischen Tenside sind die Fettalkoholethoxylate (Fettalkoholpolyglykolether, Fettalkoholoxethylate), aber auch Fettsäurealkanolamide, Alkylphenolethoxylate oder Saponine sind wichtige Gruppen der nichtionischen Tenside.

\paragraph{Amphotere Tenside}

nur wenig im wagner buch vielleicht noch wo anders schauen

\subsection{Gerüststoffe}

Gerüststoffe erfüllen in Waschmitteln viele Aufgaben. Sie haben großen Einfluss auf den Wascherfolg, unterliegen aber auch strengen ökologischen Anforderungen. Fälschlicherweise werden Gerüststoffe oft als "`Enthärter"' bezeichnet, dies ist aber nur eine der vielen Anforderungen die an moderne Gerüststoffe gestellt werden. Sie sollten außerdem eine gute Primär- und Sekundärwaschwirkung haben und gute technische Eigenschaften besitzen, d.h. gut lagerbar, verarbeitbar und verträglich mit anderen Inhaltsstoffen sein, sich nicht auf Farbe und Geruch auswirken und nicht zuletzt in großen Mengen herstellbar sein. Weiteres sind natürlich auch eine humantoxikologische und ökologische Unbedenklichkeit wichtig.

Gerüststoffe lassen sich je nach ihrer chemischen Wirkung in drei Klassen einteilen: Fällungsenthärter, Komplexbildner und Ionenaustauscher.

MEHR RECHERCHIEREN WAGNER BUCH IST CAP

\end{document}