\documentclass[main.tex]{subfiles}

\begin{document}

\subsection{Enzyme}

Enzyme sind Makromoleküle, bestehend aus einigen Hundert bis einigen Tausend Aminosäuren. Sie sind im Wesentlichen an allen biochemischen Reaktionen, als körpereigene Katalysatoren, beteiligt, von der Verdauung bis zur Vervielfältigung der DNS. Alle Enzyme enthalten ein aktives Zentrum, in welchem Stoffe gebunden und katalytisch umgesetzt werden. Durch gentechnische Veränderungen kann die Effektivität der aktiven Zentren deutlich gesteigert werden.

Für Waschmittel relevante Enzyme sind Hydrolasen, welche auch in der Verdauung eine Rolle spielen, da sie Ester, Ether, Peptide, Glycoside, Säureanhydride oder C-C-Bindungen hydrolytisch spalten. Für den Einsatz in Waschmitteln relevante Hydrolasen sind anderem:

\begin{itemize}
	\item \textbf{Proteasen:}
	Durch die Hydrolyse von Peptidbindungen kann eiweißhaltiger (FACT CHECK) Schmutz wie Gras, Schleim, Kot, Blut, Soßen, Spinat (LISTE ERWEITERN) entfernt werden.
	\item \textbf{$\alpha$-Amylasen:}
	Durch die Hydrolyse von Stärke, genauer genommen der $\alpha$-1,4-glykosidischen Polysaccharidbindungen, kann stärkehaltiger Schmutz wie Nudeln, Soßen, Fleischsaft, Pudding, Schokolade, Babynahrung oder Erdäpfeln (Kartoffeln) entfernt werden.

	https://flexikon.doccheck.com/de/Amylase
	\item \textbf{Cellulasen:}
	Durch die Hydrolyse von amorpher Cellulose, genauer genommen der entsprechenden $\beta$-1,4-D-glykosidischen Cellulosebindungen, können Baumwollfasern geglättet, aber auch die Farbkraft verbessert werden. Außerdem kommt es durch die glattere Oberfläche zu weniger Kalk- und Pigmentablagerungen.
	\item \textbf{Lipasen:}
	Durch die Hydrolyse von Triglyceriden können Fett-Verschmutzungen, wie Öle, Butter, technische Fette oder Kosmetikprodukte, bei niedrigen Temperaturen deutlich besser von Tensiden gelöst werden. Einige Lipasen verhindern auch die Ablagerung von bereits gelöstem Fett(verschmutzungen) auf der Wäsche.
	\item \textbf{Mannanasen:}
	Durch die Hydrolyse von Galactomannan, genauer genommen der $\beta$-1,4-glykosidischen Bindungen in der aus D-Mannose bestehenden Hauptkette, kann das sonst schwer entfernbare (BESSERES WORT) Polysacharid entfernt werden. Galactomannan ist Bestandteil von Guarkern- und Johannisbrotkernmehl, welche als Verdickungsmittel oder Stabilisatoren in Eiscreme, Schokolade, Ketchup und Kosmetikprodukten eingesetzt werden.

	\item Pektinase aus Englischem Enzym Paper eventuell
\end{itemize}

SPECIAL REQUIREMENTS AN ENZYME

Enzyme sind aus modernen Waschmitteln nicht mehr wegzudenken. Sie ermöglichen die hohe Leistung bei niedrigen Temperaturen von bereits 30°C (QUELLE), indem sie festsitzende Proteine, Fette und Polysaccharide hydrolysieren. Die zerkleinerten Reste können so besser durch Tenside gelöst werden.

\end{document}